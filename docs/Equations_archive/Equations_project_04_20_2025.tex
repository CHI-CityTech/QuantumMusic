\documentclass[preprintnumbers,amsmath,amssymb]{revtex4}
%\documentclass[twocolumn,showpacs,preprintnumbers]{revtex4}
%%%%%%%%%%%%%%%%%%%%%%%%%%%%%%%%%%%%%%%%%%%%%%%%%%%%%%%%%%%%%%%%%%%%%%%%%%%%%%%%%%%%%%%%%%%%%%%%%%%%%%%%%%%%%%%%%%%%%%%%%%%%%%%%%%%%%%%%%%%%%%%%%%%%%%%%%%%%%%%%%%%%%%%%%%%%%%%%%%%%%%%%%%%%%%%%%%%%%%%%%%%%%%%%%%%%%%%%%%%%%%%%%%%%%%%%%%%%%%%%%%%%%%%%%%%%
\usepackage{amssymb}
\usepackage{graphicx}
\usepackage{color}

%TCIDATA{OutputFilter=LATEX.DLL}
%TCIDATA{Version=5.50.0.2960}
%TCIDATA{<META NAME="SaveForMode" CONTENT="1">}
%TCIDATA{BibliographyScheme=Manual}
%TCIDATA{LastRevised=Thursday, August 09, 2018 19:51:30}
%TCIDATA{<META NAME="GraphicsSave" CONTENT="32">}
%TCIDATA{Language=American English}

\flushbottom \footnotesep = 0pt
\def\topfraction{1}
\def\bottomfraction{1}
\def\textfraction{0.05}
\def\floatpagefraction{0.95}
\def\no{\noindent}
\def\bc{\begin{center}}
\def\ec{\end{center}}
\def\vs{\vskip0.5cm}
\def\beq{\begin{equation}}
\def\eeq{\end{equation}}
\def\cl{\centerline}
\def\d{\downarrow}
\def\u{\uparrow}
\def\doub{\downarrow\uparrow}
\def\bj{{\bf j}}
\def\bl{{\bf k}}
\def\br{{\bf r}}
\def\bq{{\bf q}}
\def\bk{{\bf k}}
\textfloatsep = 0.5cm \floatsep = 0.0cm \setcounter{topnumber}{2}
\setcounter{bottomnumber}{2}

\begin{document}
\title{Model Equations for the music project}

\author{}








 \maketitle

\section{Model Equations}


We study $N$ musical compositions ($i = 1,2, \ldots, N$) at the time
intervals for each musical composition $t$: $0  \leq t_{i} \leq
T_{i}$.

Then we divide the maximal time interval with the maximal $T_{i}$
into $M$ small parts.

Then we have the time steps for each musical composition $i$: $(0,
t_{i1})$, $(t_{i1}, t_{i2})$, \ldots,  $(t_{i(M-1)}, t_{iM})$.

We represent each $i$th musical composition as a trajectory is the
three-dimensional space $r_{i} = r_{i}(t)$, which can be represented as $x_{i} =
x_{i}(t)$, $y_{i} = y_{i}(t)$, $z_{i} = z_{i}(t)$. We assume $r_{i}(t) = (x_{i}(t), y_{i}(t),
z_{i}(t))$.

We assume that we have $L$ characteristic properties of musical composition: $C_{l}$; where $l =
1, \ldots, L$.

In this case, each $j$th point in the $i$th trajectory, corresponding to the $i$th
musical composition, can be represented as
\begin{eqnarray}
\label{traj} x_{i}(t_{ij}) = \sum_{l = 1}^{L} p_{ilx}(t_{ij}) C_{l}; \hspace{2cm} y_{i}(t) = \sum_{l = 1}^{L} p_{ily}(t_{ij}) C_{l};
 \hspace{2cm} z_{i}(t) = \sum_{l = 1}^{L} p_{ilz}(t_{ij}) C_{l} ,
\end{eqnarray}
where $j = 1,2, \ldots, M$. The coefficients $0 \leq p_{ilx}(t_{ij}) \leq 1$, $0 \leq p_{ily}(t_{ij}) \leq 1$, $0 \leq p_{ilz}(t_{ij}) \leq 1$ represent
the relative contribution of the characteristic property $C_{l}$ into the components $x$, $y$, and $z$ correspondingly for the musical composition $i$ during the time intervals
$(t_{i(j-1)}, t_{ij})$. The coefficients $0 \leq p_{ilx}(t_{ij}) \leq 1$, $0 \leq p_{ily}(t_{ij}) \leq 1$, $0 \leq p_{ilz}(t_{ij}) \leq 1$ have to be defined
from the known $N$ compositions.

In this case we generate the new generated musical composition $R$, represented by the following trajectory:
\begin{eqnarray}
\label{traj1} x_{R}(t_{j}) = \sum_{l = 1}^{L} P_{Rlx}(t_{Rj}) C_{l}; \hspace{2cm} y_{R}(t) = \sum_{l = 1}^{L} P_{Rjy}(t_{Rj}) C_{l};
 \hspace{2cm} z_{R}(t) = \sum_{l = 1}^{L} P_{Rlz}(t_{Rj}) C_{l} ,
\end{eqnarray}
where   $P_{Rlx}(t_{Rj})$, $P_{Rly}(t_{Rj})$, $P_{Rlz}(t_{Rj})$  are the expectations of the coefficients
$p_{ilx}(t_{ij})$, $p_{ily}(t_{ij})$, $p_{ilz}(t_{ij})$ for each characteristic property $C_{l}$ during the time interval $(t_{R(j-1)}, t_{Rj})$,
for $x$, $y$, and $z$ components of the trajectory for the new generated musical composition $R$.
The expectations $P_{Rlx}(t_{Rj})$, $P_{Rly}(t_{Rj})$, $P_{Rlz}(t_{Rj})$ are given by
\begin{eqnarray}
\label{prob} P_{Rlx}(t_{Rj}) = \frac{\sum_{i = 1}^{N}  p_{ilx}(t_{ij})}{N}; \hspace{2cm}
P_{Rly}(t_{Rj}) = \frac{\sum_{i = 1}^{N}  p_{ily}(t_{ij})}{N}; \hspace{2cm}
P_{Rlz}(t_{Rj}) = \frac{\sum_{i = 1}^{N}  p_{ilz}(t_{ij})}{N} .
\end{eqnarray}

We can use the quantum computer in order to generate new musical compositions, represented by new random trajectories in the three-dimensional ($x$, $y$, $z$) space,
by defining the new trajectories by random processes using Eq.~(\ref{traj1}) and assuming that $P_{Rlx}(t_{Rj})$, $P_{Rly}(t_{Rj})$, $P_{Rlz}(t_{Rj})$  are the probabilities that each characteristic property $C_{l}$ can be represented
in the time interval $(t_{R(j-1)}, t_{Rj})$, for $x$, $y$, and $z$ components of the trajectory for the new generated musical composition $R$.





\end{document}




















Since the concept of quantum entanglement was introduced, quantum
entanglement has been receiving more and more
attention~\cite{Nielsen,Horodecki} Specifically in recent years,
with the interdisciplinary development of quantum informatics, the
general definition, physical properties and entanglement measurement
of quantum entanglement have been comprehensively and deeply
studied. Due to its nonlocality, quantum entanglement has been
widely used in quantum information, especially in quantum computing
and quantum communication~\cite{Zou}.

Quantum entanglement, particularly with photons, is crucial for
quantum communication, as it allows for secure communication and the
sharing of quantum information.

Entanglement is a fundamental quantum phenomenon where two or more
particles become linked in such a way that they share the same fate,
no matter how far apart they are. This linked state allows for the
transfer of information and the creation of new quantum states.
Photons, being light particles, are ideal for quantum communication
because they can travel long distances and retain their quantum
properties.

Entangled photons can be used for quantum key distribution (QKD),
which provides a secure way to share encryption keys. Entanglement
enables the transfer of quantum information from one photon to
another, even if they are far apart. Entangled photons are essential
for building quantum networks, which can facilitate the transfer of
quantum information between different quantum processors.

In this Letter we provide an approach to monitor the entanglement of
photons, which is based on a measurement protocol. We consider a
photonic cavity with one and two two-level qubits inside. Then the
cavity photons become entangled through their interaction with these
qubits. The resulting entanglement is controlled by repeated
projective measurements, which are periodically repeated with the
time step $\tau$.

We  study the quantum entanglement created between cavity photons
coupled to one and two qubits applying a Jaynes-Cummings model. We
employ the Hamiltonian used for the Jaynes-Cummings
model~\cite{Jaynes,Knight} for one and two qubits, coupled to cavity
photons.


As a measure of quantum entanglement we use the R\'{e}nyi
entanglement entropy, which  measures the quantum correlations
between two subsystems under a spatial
bipartition~\cite{Bombelli,Srednicki,Eisert,Miao}, which has been
applied for detecting measurement-induced entanglement
transitions~\cite{Li,Skinner,Lunt}. Recently, the R\'{e}nyi
entanglement entropy has been efficiently measured in the
experiment~\cite{vanEnk,Elben,Satzinger}.







\begin{thebibliography}{99}

\bibitem{Nielsen} M.~A. Nielsen and I.~L. Chuang,  {\it Quantum Computation and Quantum
Information}, (Cambridge Univ. Press, 2000).

\bibitem{Horodecki}  R. Horodecki, P. Horodecki, M. Horodecki, and K.
Horodecki, \rmp {\bf 81}, 865 (2009).


\bibitem{Zou} N. Zou, J. Phys.: Conf. Ser. {\bf 1827} 012120 (2021).


\bibitem{Jaynes} E.~T. Jaynes and F.~W. Cummings, Proc. IEEE {\bf 51}, 89 (1963).

\bibitem{Knight} B.~W. Shore and P.~L. Knight, J.~Mod.~Opt. {\bf 40}, 1195 (1993).

\bibitem{Bombelli} L. Bombelli, R. K. Koul, J. Lee, and R. D. Sorkin, prd {\bf 34}, 373 (1986).

\bibitem{Srednicki}  M. Srednicki, \prl {\bf 71}, 666 (1993).

\bibitem{Eisert} J. Eisert, M. Cramer, and M. B. Plenio, \rmp {\bf 82}, 277
(2010).

\bibitem{Miao}  Q. Miao and T. Barthel, \prl {\bf 127}, 040603 (2021).


\bibitem{Li}  Y. Li, X. Chen, and M. P. A. Fisher, \prb {\bf 100}, 134306
(2019).

\bibitem{Skinner} B. Skinner, J. Ruhman, and A. Nahum, Phys.~Rev.~X {\bf 9}, 031009
(2019).

\bibitem{Lunt} O. Lunt, M. Szyniszewski, and A. Pal, \prb {\bf 104},
155111 (2021).

\bibitem{vanEnk} S. J. van Enk and C. W. J. Beenakker, \prl {\bf 108},
110503 (2012).

\bibitem{Elben} A. Elben, B. Vermersch, M. Dalmonte, J. I. Cirac, and P.
Zoller, \prl {\bf 120}, 050406 (2018).

\bibitem{Satzinger} K. J. Satzinger et al., Science {\bf 374}, 1237 (2021).


\end{thebibliography}

\end{document}

\section{A JC-like model for two qubits in a cavity}
\label{sec3}


Qubits form the basics for quantum computing, and quantum
entanglement is the key resource present in quantum computing that
makes it such a powerful tool. Under these conditions, we applied a
Jaynes-Cummings-like model to study the quantum entanglement created
between two qubits that are coupled to a single cavity mode.
\medskip
\par

Let us consider a system consisting of two qubits interacting with a
cavity mode with Hamiltonian

\begin{equation}
\hat{H}=\hat{H}_0+\hat{V^\prime}_{RWA}\label{hami}
\end{equation}
with  the unperturbed Hamiltonian $\hat{H}_0$  given by

\begin{equation}
\hat{H}_0=\hbar\left(\sum_{j=1}^2\omega_0\ket{e_j}\bra{e_j}+\omega_k\hat{a}^\dagger\hat{a}\right)\label{H0}
\   ,
\end{equation}
where, in this notation,  $\ket{e_j}$ is  the excited state of qubit
$j$, $\hbar\omega_0$ is the energy gap of the qubits, $\omega_k$ is
the mode frequency and $\hat{a}^\dagger$ and $\hat{a}$ are the
creation and annihilation operators for photons in the cavity. Here,
we consider the excited state for the qubit, $\ket{e}$, to consist
of a qubit in its first excited state, and $\ket{g}$ to consist of
that same qubit in the ground state. Also, for simplicity, in Eq.\
(\ref{H0}), we relabel the qubit energies so that the energy of a
qubit in the ground state is set equal to 0. The energy gap
$\hbar\omega_0$ is equal to $\hbar\omega_0 =
E^\prime_{0,1}-E^\prime_{0,0}$. Therefore, the eigenstates of
$\hat{H_0}$ are $\ket{n;ij}$, having energy $E_{nij}$ given by

\begin{equation}
E_{nij} = \hbar [\omega_0(i + j) + n\omega_k]
\end{equation}
where $i = 1$ ($j = 1$) if the first (second) qubit is in the
excited state $\ket{e}$ and $i = 0$ ($j = 0$) if the first (second)
qubit is in the ground state $\ket{g}$.

\medskip
\par

The interaction Hamiltonian on the rotating wave approximation
(RWA), $\hat{V^\prime}_{RWA}$ is given by


\begin{equation}
\hat{V^\prime}_{RWA} =
\hbar\lambda\sum_{j=1}^2\left(\hat{\sigma}_j^+\hat{a}+\hat{\sigma}_j^-\hat{a}^\dagger\right)
\end{equation}
where $\hat{\sigma}_j^\pm$ are the creation (+) and annihilation (-)
operators for the qubit $j$ and $\lambda$ is the qubit-photon
coupling constant.

\medskip
\par


This Hamiltonian is similar to that for     the Jaynes-Cummings
model, except that it is for two qubits in one cavity mode. Since
the qubits interact with the mode according to the RWA, the
destruction (creation) of photons is determined by the creation
(annihilation) of qubit excitations. Therefore,  as for the
Jaynes-Cummings model, we can treat each manifold with fixed number
of excitations, meaning the sum of the number of photons and the
number of qubit excitations, individually. The $n$-th manifold is
composed of the states with $n$ total excitations, namely
$\ket{n;00}\equiv\ket{0_n}$, which is the state with $n$ cavity
photons and no qubit excitations; $\ket{n-1;01}\equiv\ket{1_n}$, the
state with $n-1$ cavity photons and with  qubit ``1" in the excited
state $\ket{e}$; $\ket{n-1;10}\equiv\ket{2_n}$,  the state with
$n-1$ cavity photons and with the qubit ``2" in the excited state
$\ket{e}$ and $\ket{n-2;11}\equiv \ket{3_n}$, the state with $n-2$
total excitations and both qubits in the excited state. It is
important to note that each manifold is 4-dimensional with exception
to the $n=1$, which is 3-dimensional and to the $n=0$ which is
composed only with  the ground  state, $\ket{0;00}$. The effective
Hamiltonian, $\hat{H}_n$, on the $n$-th manifold appears directly
from Eq.\ (\ref{hami}) and is equal to

\begin{equation}
H_1 = \hbar \begin{pmatrix}
\omega_k &\lambda&\lambda\\
\lambda & \omega_0 &0 \\
\lambda & 0 & \omega_0
\end{pmatrix},\label{H1}
\end{equation}
for $n=1$ and

\begin{equation}
H_n = \hbar \begin{pmatrix}
n\omega_k &\lambda\sqrt{n}&\lambda\sqrt{n}& 0\\
\lambda\sqrt{n} & (n-1)\omega_k+\omega_0 &0 & \lambda\sqrt{n-1}\\
\lambda\sqrt{n} & 0 & (n-1)\omega_k+\omega_0 &\lambda\sqrt{n-1} \\
0 & \lambda\sqrt{n-1}& \lambda\sqrt{n-1}&(n-2)\omega_k+2\omega_0
\end{pmatrix},\label{Hn}
\end{equation}
when $n=2,3,\cdots$. For the ground state, $H\ket{0;00}=0$. One can
find in a straightforward way the energy eigenstates for the
Hamiltonian in (\ref{H1}). The eigenvalues of (\ref{H1}) are

\begin{eqnarray}
\epsilon_0 &=& \omega_0\label{l11}
\nonumber\\
\epsilon_\pm &=&
\dfrac{1}{2}\left((\omega_0+\omega_k)\pm\sqrt{(\omega_0-\omega_k)^2+{8\lambda^2}}\right),\label{l12}
\end{eqnarray}
with eigenvectors

\begin{eqnarray}
\ket{\psi_0} &=&
\dfrac{1}{\sqrt{2}}\left(\ket{0;01}-\ket{0;10}\right)\label{vec11}
\nonumber\\
\ket{\psi_\pm} &=&
\dfrac{1}{\sqrt{2+a_\pm^2}}\left(a_\pm\ket{1;00}-\ket{0;01}-\ket{0;10}\right),\label{vect12}
\end{eqnarray}
where $a_\pm$ is given by

\begin{equation}
a_\pm =
\dfrac{1}{2\lambda}\left[(\omega_0-\omega_k)\pm\sqrt{(\omega_0-\omega_k)^2+{8\lambda^2}}\right].\label{apm}
\end{equation}


\medskip
\par

The eigenvalues and eigenvectors of (\ref{Hn}), however, are much
more difficult to obtain.   However, this problem is solved  when we
consider the resonance case where the cavity frequency is the same
as the qubit transition frequency, $\omega_0=\omega_k \equiv
\omega$. In this case, Eqs.\ (\ref{H1}) and (\ref{Hn}) become


\begin{eqnarray}
H_1 &=& \hbar \begin{pmatrix}
\omega &\lambda&\lambda\\
\lambda & \omega &0 \\
\lambda & 0 & \omega
\end{pmatrix}\label{H1res}
\\
H_n &=& \hbar \begin{pmatrix}
n\omega &\lambda\sqrt{n}&\lambda\sqrt{n}& 0\\
\lambda\sqrt{n} & n\omega &0 & \lambda\sqrt{n-1}\\
\lambda\sqrt{n} & 0 & n\omega &\lambda\sqrt{n-1} \\
0 & \lambda\sqrt{n-1}& \lambda\sqrt{n-1}&n\omega
\end{pmatrix}\label{Hnres}. \  .
\end{eqnarray}
The eigenvalues of (\ref{H1res}) are

\begin{eqnarray}
\epsilon_0 &=& \omega
\nonumber\\
\epsilon_\pm &=& \omega\pm \sqrt{2}\lambda,
\end{eqnarray}
with eigenvectors

\begin{eqnarray}
\ket{\psi_0} &=&
\dfrac{1}{\sqrt{2}}\left(\ket{0;01}-\ket{0;10}\right)\label{eg11}
\\
\ket{\psi_\pm} &=&
\dfrac{1}{2}\left(\pm\sqrt{2}\ket{1;00}-\ket{0;01}-\ket{0;10}\right)\label{eg12},
\end{eqnarray}
which are  simply Eqs.\ (\ref{l11}-\ref{vect12}) at resonance.


\medskip
\par



The eigenvalues of Eq.\ (\ref{Hnres}) are given by

\begin{eqnarray}
{\epsilon_n}_0 &=& n\omega
\\
{\epsilon_n}_\pm &=& n\omega\pm\sqrt{2(2n-1)}\lambda  \  ,
\end{eqnarray}
where the first eigenvalue is doubly degenerate. The corresponding
eigenvectors are as follows

\begin{eqnarray}
\ket{{\psi_n}_{0,1}} &=&
\dfrac{1}{\sqrt{2}}(\ket{n-1;01}-\ket{n-1;10})\label{egn1}
\\
\ket{{\psi_n}_{0,2}}
&=&\dfrac{1}{\sqrt{2n-1}}\left(\sqrt{n-1}\ket{n;00}-\sqrt{n}\ket{n-2;11}\right)\label{psi_2}
\\
\ket{{\psi_n}_\pm} &=&
\dfrac{1}{\sqrt{8n-4}}\left[\sqrt{2n}\ket{n;00}+\sqrt{2(n-1)}\ket{n-2;11}\pm\sqrt{2n-1}\left(\ket{n-1;01}+\ket{n-1;10}\right)\right]\
.
%\nonumber
\label{egnl}
\end{eqnarray}
Here,  $\ket{{\Psi_n}_{0,1}}$ and $\ket{{\Psi_n}_{0,2}}$ are the two
degenerate eigenstates with eigenvalue $\epsilon_{n0}$ and
$\ket{{\psi_n}_\pm}$ are the eigenstates corresponding to the
eigenvalue $\epsilon\pm$.    It is interesting to note that the
eigenvalues and eigenvectors of Eqs.\ (\ref{H1res}) and
(\ref{Hnres}) do agree with each other for the case $n=1$,  with the
exception of the eigenvector $\ket{{\psi_n}_{0,2}}$, which, for
$n=1$, contains a vector that does not exist in reality,
corresponding to a state with 1 photon, as can be readily seen from
Eq.\ (\ref{psi_2}).

\end{document}
