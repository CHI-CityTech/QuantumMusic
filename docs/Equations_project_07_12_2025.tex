\documentclass[preprintnumbers,amsmath,amssymb]{revtex4}
%\documentclass[twocolumn,showpacs,preprintnumbers]{revtex4}
%%%%%%%%%%%%%%%%%%%%%%%%%%%%%%%%%%%%%%%%%%%%%%%%%%%%%%%%%%%%%%%%%%%%%%%%%%%%%%%%%%%%%%%%%%%%%%%%%%%%%%%%%%%%%%%%%%%%%%%%%%%%%%%%%%%%%%%%%%%%%%%%%%%%%%%%%%%%%%%%%%%%%%%%%%%%%%%%%%%%%%%%%%%%%%%%%%%%%%%%%%%%%%%%%%%%%%%%%%%%%%%%%%%%%%%%%%%%%%%%%%%%%%%%%%%%
\usepackage{amssymb}
\usepackage{graphicx}
\usepackage{color}

%TCIDATA{OutputFilter=LATEX.DLL}
%TCIDATA{Version=5.50.0.2960}
%TCIDATA{<META NAME="SaveForMode" CONTENT="1">}
%TCIDATA{BibliographyScheme=Manual}
%TCIDATA{LastRevised=Thursday, August 09, 2018 19:51:30}
%TCIDATA{<META NAME="GraphicsSave" CONTENT="32">}
%TCIDATA{Language=American English}

\flushbottom \footnotesep = 0pt
\def\topfraction{1}
\def\bottomfraction{1}
\def\textfraction{0.05}
\def\floatpagefraction{0.95}
\def\no{\noindent}
\def\bc{\begin{center}}
\def\ec{\end{center}}
\def\vs{\vskip0.5cm}
\def\beq{\begin{equation}}
\def\eeq{\end{equation}}
\def\cl{\centerline}
\def\d{\downarrow}
\def\u{\uparrow}
\def\doub{\downarrow\uparrow}
\def\bj{{\bf j}}
\def\bl{{\bf k}}
\def\br{{\bf r}}
\def\bq{{\bf q}}
\def\bk{{\bf k}}
\textfloatsep = 0.5cm \floatsep = 0.0cm \setcounter{topnumber}{2}
\setcounter{bottomnumber}{2}

\begin{document}
\title{Simplified Model Equations for the music project}

\author{}








 \maketitle

\section{Simplified Model Equations}


We study $N$ musical compositions ($i = 1,2, \ldots, N$) at the time
intervals for each musical composition $t$: $0  \leq t_{ij} \leq
T_{i}$.

Then we divide the maximal time interval with the maximal $T_{i}$
into $M$ small parts. For any musical composition $i$ we do not
consider $t_{ij}$ at $j > M$.


Then we have the time steps for each musical composition $i$: $(0,
t_{1})$, $(t_{1}, t_{2})$, \ldots,  $(t_{(M-1)}, t_{M})$.  We
emphasize that $t_{M} = T$.

We represent each $i$th musical composition as a trajectory is the
three-dimensional space $\vec{r}_{i} = \vec{r}_{i}(t)$, which can be
represented as $x_{i} = x_{i}(t)$, $y_{i} = y_{i}(t)$, $z_{i} =
z_{i}(t)$. We assume $\vec{r}_{i}(t) = (x_{i}(t), y_{i}(t),
z_{i}(t))$.


Then we calculate the expectations of $x$, $y$, and $z$ at the time
step $t_{j}$ as
\begin{eqnarray}
\label{expect} \bar{x}_{j}(t_{j}) = \frac{1}{N} \sum_{i = 1}^{N}
x_{i}(t_{ij}); \hspace{2cm} \bar{y}_{j}(t_{j}) = \frac{1}{N} \sum_{i
= 1}^{N} y_{i}(t_{ij});
 \hspace{2cm} \bar{z}_{j}(t_{j}) = \frac{1}{N} \sum_{i
= 1}^{N} z_{i}(t_{ij}),
\end{eqnarray}
where $j = 1,2, \ldots, M$.

We define the standard deviations $\sigma (x)$, $\sigma (y)$,and
$\sigma (z)$ for $x$, $y$, and $z$, respectively,  at the time step
$t_{j}$ as
\begin{eqnarray}
\label{sigma} && \sigma_{j} (x) = \frac{1}{\sqrt{N}} \left[ \sum_{i
= 1}^{N} \left(x_{i}(t_{ij}) -  \bar{x}_{j}(t_{j})
\right)^{2}\right]^{1/2} ; \hspace{0.5cm}
 \sigma_{j} (y) = \frac{1}{\sqrt{N}} \left[ \sum_{i =
1}^{N} \left(y_{i}(t_{ij}) -  \bar{y}_{j}(t_{j})
\right)^{2}\right]^{1/2} ; \nonumber \\
&& \sigma_{j} (z) = \frac{1}{\sqrt{N}} \left[ \sum_{i = 1}^{N}
\left(z_{i}(t_{ij}) -  \bar{z}_{j}(t_{j}) \right)^{2}\right]^{1/2} ,
\end{eqnarray}
where $j = 1,2, \ldots, M$.


We can use the quantum computer in order to generate new musical
compositions, represented by new random trajectories in the
three-dimensional ($x$, $y$, $z$) space, by defining the new
trajectories by random processes using the normal (Gaussian)
distribution at the time step $t_{j}$ with the probability density
function for $x$, $y$, and $z$, respectively, defined as
\begin{eqnarray}
\label{f} && f(x_{j}) =  \frac{1}{\sqrt{2\pi \sigma_{j}^{2}(x)}}
\exp\left[-\frac{\left(x_{j}(t_{j}) - \bar{x}_{j}(t_{j})
\right)^{2}}{2\sigma_{j}^{2}(x)}\right] ; \hspace{0.5cm} f(y_{j}) =
\frac{1}{\sqrt{2\pi \sigma_{j}^{2}(y)}}
\exp\left[-\frac{\left(y_{j}(t_{j}) -  \bar{y}_{j}(t_{j})
\right)^{2}}{2\sigma_{j}^{2}(y)}\right] ; \nonumber \\
&& f(z_{j}) = \frac{1}{\sqrt{2\pi \sigma_{j}^{2}(z)}}
\exp\left[-\frac{\left(z_{j}(t_{j}) -
\bar{z}_{j}(t_{j})\right)^{2}}{2\sigma_{j}^{2}(z)}\right] ,
\end{eqnarray}
where $j = 1,2, \ldots, M$.

In this theory we applied the central limit theorem, which states
that under certain (fairly common) conditions, the sum of many
random variables will have an approximately normal distribution.

\end{document}
